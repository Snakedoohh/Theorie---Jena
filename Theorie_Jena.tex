%-----Config---
\documentclass[BCOR=6mm,headinclude=true,footinclude=false,12pt,chapterprefix,numbers=noendperiod,twoside,a4paper,openright,titlepage]{scrreprt}%chapterprefix macht, dass "`Kapitel Nr"' über Kapitelüberschrift steht, numbers=noendperiod macht, dass hinter der letzten Nummer kein Punkt steht, BCOR ist die Bindekorrektur, also das, was am Innenrand zusätzlich frei gelassen wird, DVI macht Seitenränder größer oder kleiner, headinclude/footinclude zählt Kopf/Fußzeilen zum Textkörper oder eben nicht
\usepackage{fontspec}
\defaultfontfeatures{Mapping=tex-text}
%\setmainfont[SmallCapsFont = Fontin SmallCaps]{Fontin}
\setmainfont{Times New Roman}
\usepackage{fullpage}
%\raggedbottom%macht dass die letzte Zeile da liegt wo sie eben liegt und nicht auf jeder Seite gleich (sonst werden Absatzabstände gedehnt)
\usepackage{array,amsmath,amssymb,amstext}
\usepackage{graphicx,subfig,longtable,makeidx}
\usepackage[section]{placeins} % verhindert gleiten der Bilder  der 
%section
\usepackage{float}
\usepackage{wrapfig} %für textumflossene grafiken: \begin{wrapfigure}{l}{8.5cm} l für links, r für rechts, cm für wie groß die Box sein soll
\usepackage{sidecap}%Packet für siedecaption, dann statt \begin{figure} \begin{SCfigure}[1] schreiben
\usepackage{pdfpages}
\usepackage[colorlinks,pdfpagelabels, pdfstartview = FitH, bookmarksopen = true, bookmarksnumbered = true, linkcolor = black, plainpages = false, hypertexnames = false, citecolor = black]{hyperref}
\usepackage{url}
\usepackage{fixltx2e}
\usepackage{polyglossia}
\setdefaultlanguage{german}
\usepackage[font=it,labelfont=it]{caption}
%\setcounter{secnumdepth}{2} % macht das nicht subsubsections nummeriert werden
\setcapindent{0pt}
%\sidecaptionvpos{figure}{c}%macht, dass Sidecaptions in der Mitte der Höhe der Grafik
%\setcapindent{0pt} %kein einzug in Bildunterschrifte
\newcommand{\vektor}[1]{\textbf{\itshape{#1}}} %newcommands verwenden um einheitliche Vektoren zu bekommen
\renewcommand{\thefootnote}{}%macht, dass Fußnoten nicht nummeriert sind
\usepackage[headsepline,automark]{scrpage2}%scrpage2 ist Packet für Kopf/Fußzeilen, headsepline ist Linie unter Kopfzeile, automark aktualisiert automatisch die Titel in Kopfzeile
\usepackage[onehalfspacing]{setspace}
%\unitlength1cm
%\renewcommand{\baselinestretch} {1.3}
\pagestyle{scrheadings}
\clearscrheadfoot
\ofoot[\pagemark]{\pagemark}
\ihead{\headmark}
\setlength{\textfloatsep}{1cm}
%\renewcommand*{\chapterpagestyle}{chapter} 
\graphicspath{{./Bilder/}}
\setlength{\headsep}{1cm}
\addtokomafont{sectioning}{\rmfamily}
\date{\today}
\begin{document}
\hyphenation{mag-net-isch Mag-net-feld}
\hyphenation{mi-cros-co-py}
%------------------------------------- Titelseite--------------

%\maketitle
\begin{titlepage}
\title{\vspace{-1.5cm}
Theoretische Modelle zum Pinningverhalten von Hochtemperatursupraleitern - von empirischen Modellen bis zur Quanetenmechanik
\ \\
\ \\}

\subtitle{
%\includegraphics[width=0.45\textwidth]{Front_1.png}
\ \\
\ \\
\Large
	\ \\
	\Large
	\ \\
	\Large
		\ \\
	\Large 
	\author{\textbf{Stephen Ruoß}}
	\vspace{-0.8cm}}
%\author{Stephen Ruoß}
\date{}

\uppertitleback{}
\pagenumbering{Roman}
%\setcounter{page}{3}
\maketitle
\newpage
\end{titlepage}

%---------Inhaltsverzeichnis-----------------------------------
\thispagestyle{empty}
\section*{}
\newpage
\pagenumbering{gobble}
\include{abstract}
\thispagestyle{empty}
\section*{}
\newpage
\pagenumbering{Roman}
\tableofcontents

%---------Einleitung-------------------------------------------
\newpage
\thispagestyle{empty}
\pagenumbering{gobble}
\section*{}
\newpage
\pagenumbering{arabic}
\include{Einleitung}

%---------Theorie----------------------------------------------



%\newpage
%\includepdf[pages=-]{WordDocs/selbst.pdf}

\end{document}